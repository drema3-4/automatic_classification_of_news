\documentclass[autoref]{SCWorks}
\usepackage[T2A]{fontenc}
\usepackage[utf8]{inputenc}
\usepackage{graphicx}
\usepackage[sort,compress]{cite}
\usepackage{amsmath}
\usepackage{amssymb}
\usepackage{amsthm}
\usepackage{fancyvrb}
\usepackage{longtable}
\usepackage{array}
\usepackage[english,russian]{babel}
\usepackage{minted}
% Используется автором репозитория
\usemintedstyle{xcode}
\usepackage[colorlinks=false]{hyperref}

\usepackage{tempora}
\usepackage{cmap}
% ==============================================================================
\newcommand{\eqdef}{\stackrel {\rm def}{=}}
\newtheorem{lem}{Лемма}
\hypersetup{
colorlinks,
citecolor=black,
filecolor=black,
linkcolor=black,
urlcolor=black
}
\AtBeginEnvironment{minted}{\renewcommand{\fcolorbox}[4][]{#4}}

\newcommand{\mintcode}[2]{
    \usemintedstyle{vs}
    \inputminted[fontsize=#1, encoding=utf8, outencoding=utf8]{python}{#2}
}
% ==============================================================================
\begin{document}

% Кафедра (в родительном падеже)
\chair{математической кибернетики и компьютерных наук}

% Тема работы
\title{Автоматическая тематическая классификация новостного массива}

% Курс
\course{4}

% Группа
\group{451}

% Факультет (в родительном падеже) (по умолчанию "факультета КНиИТ")
% \department{факультета КНиИТ}

% Специальность/направление код - наименование
\napravlenie{09.03.04 "--- Программная инженерия}

% Фамилия, имя, отчество в родительном падеже
\author{Кондрашова Даниила Владиславовича}

% Заведующий кафедрой 
\chtitle{доцент, к.\,ф.-м.\,н.}
\chname{С.\,В.\,Миронов}

% Научный руководитель
\satitle{доцент, к.\,ф.-м.\,н.} % должность, степень, звание
\saname{С.\,В.\,Папшев}

% Руководитель практики от организации (руководитель для цифровой кафедры)
\patitle{доцент, к.\,ф.-м.\,н.}
\paname{С.\,В.\,Миронов}

% Год выполнения отчета
\date{2025}
% ==============================================================================
\maketitle
% Включение нумерации рисунков, формул и таблиц по разделам (по умолчанию -
% нумерация сквозная) (допускается оба вида нумерации)
\secNumbering
% ==============================================================================
% Введение
%%%%%%%%%%%%%%%%%%%%%%%%%%%%%%%%%%%%%%%%%%%%%%%%%%%%%%%%%%%%%%%%%%%%%%%%%%%%
%%%%%%%%%%%%%%%%%%%%%%%%%%%%%%%%%%%%%%%%%%%%%%%%%%%%%%%%%%%%%%%%%%%%%%%%%%%%
%%%%%%%%%%%%%%%%%%%%%%%%%%%%%%%%%%%%%%%%%%%%%%%%%%%%%%%%%%%%%%%%%%%%%%%%%%%%
%%%%%%%%%%%%%%%%%%%%%%%%%%%%%%%%%%%%%%%%%%%%%%%%%%%%%%%%%%%%%%%%%%%%%%%%%%%%
\intro
В настоящее время оперативный поиск информации становится критически важной
задачей. Однако анализ полного массива данных невозможен из-за его масштабов,
что создаёт необходимость в классификации и последующей фильтрации данных для
выделения релевантной информации. Решением этой проблемы может служить
тематическая классификация.

Большие объёмы данных, такие как новостные потоки, часто не имеют системной
тематической разметки. Даже при наличии рубрикации, её субъективность может
приводить к проблемам: некорректному присвоению тем, избыточности тематических
категорий и их недостаточному охвату. Это вызывает ошибки при поиске и анализе
информации. Для устранения этих недостатков требуется механизм, обеспечивающий
точную тематическую классификацию с возможностью автоматической разметки
новостных материалов.

Одним из инструментов для реализации такого подхода являются тематические
модели в сочетании с алгоритмами глубокого обучения. Первые позволяют выявить
скрытые темы в текстовых данных и подготовить разметку для обучения вторых.
Алгоритмы глубокого обучения, в свою очередь, могут классифицировать новые
тексты по заданным темам.

Таким образом, целью данной работы является разработка нейросетевого метода
автоматической классификации новостей на основе тематической модели предметной
области.

Для достижения цели необходимо решить следующие задачи:
\begin{enumerate}
    \item Выполнить парсинг новостных данных и их текстовую предобработку;
    \item Провести анализ характеристик и параметров набора данных;
    \item Выполнить тематическое моделирование подготовленных данных с
    оптимальными параметрами;
    \item Разметить данные для обучения нейронной сети"=классификатора с
    помощью тематического моделирования;
    \item Выполнить обучение нейронной сети"=классификатора на размеченных
    данных;
    \item Провести анализ качетсва обученной модели;
    \item Проанализировать эффективность разработанного метода автоматической
    тематической классификации.
\end{enumerate}

%%%%%%%%%%%%%%%%%%%%%%%%%%%%%%%%%%%%%%%%%%%%%%%%%%%%%%%%%%%%%%%%%%%%%%%%%%%%
%%%%%%%%%%%%%%%%%%%%%%%%%%%%%%%%%%%%%%%%%%%%%%%%%%%%%%%%%%%%%%%%%%%%%%%%%%%%
%%%%%%%%%%%%%%%%%%%%%%%%%%%%%%%%%%%%%%%%%%%%%%%%%%%%%%%%%%%%%%%%%%%%%%%%%%%%
%%%%%%%%%%%%%%%%%%%%%%%%%%%%%%%%%%%%%%%%%%%%%%%%%%%%%%%%%%%%%%%%%%%%%%%%%%%%
\conclusion
В ходе данной дипломной работы был разработан алгоритм автоматической
классификации новостей на основе тематической модели предметной области.

Для этого было выполнено следующее:

\begin{enumerate}
    \item Проведён анализ инструментов по сбору данных и выбраны
    наиболее удобные из них (BeautifulSoup4, requests);
    \item Проведён сбор данных;
    \item Проанализированы способы обработки текстовых данных
    и выбранны наиболее удобные из них;
    \item Проанализированы популярные инструменты для обработки
    текстовых данных (NLTK, Pymorphy3, SpaCy) и выбран наиболее
    удобный и точный из них (SpaCy);
    \item Проведена подготовка данных для тематического моделирования и
    проведён анализ её результатов;
    \item Изучен механизм тематического моделирования с помошью
    аддитивной тематической регуляризации;
    \item Разработаны инструменты для тематической классификации с
    помощью библиотеки BigARTM;
    \item Проведены эксперименты по проведению тематической классификации
    над подготовленными различными способами данными, а также
    проведён анализ результатов экспериментов;
    \item Рассмотрены различные способы обработки текстовых данных нейронными
    сетями и выбран наиболее подходящий из них (семантическое векторное
    представление);
    \item Проведён анализ архитектур подходящих типов нейронных сетей
    и выбрана наиболее подходящая из них (transformer);
    \item Проведён анализ доступных предобученных сетей и сервисов, которые
    их предоставляют, в ходе которого выбран наиболее удобный из них (Hugging
    Face и Roberta);
    \item Проведены эксперименты по обучению тематического классификатора
    новостей, а также выполнен анализ результатов и сделаны соответствующие
    выводы.
\end{enumerate}

Основной вывод по итогам работы: предложенный метод автоматической классификации
имеет перспективу применения при более тщательном тематическом моделировании
исходного набора данных.

\end{document}